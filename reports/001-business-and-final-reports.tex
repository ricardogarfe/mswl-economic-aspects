\documentclass[11pt]{scrartcl}
\usepackage[parfill]{parskip}
\usepackage{graphicx}
\usepackage{booktabs}
\usepackage{tabulary}
\usepackage{float}

\title{\textbf{Economic Aspects\\
                Lesson 4}}
\subtitle{Open Core Business Model}
\author{Jesús M. González-Barahona}
\date{\today}

\begin{document}

\maketitle

\section{Business model}

Minimum: 20 points, maximum: 80 points
Business plan for a company in which libre software clearly has an impact. The plan has to be detailed enough. 

Detailed description:

Think about an hypothetical start-up company, with an activity based libre software, or for which libre software may mean a difference. Create a short business plan for it.

A previous discussion can be done in groups of up to three persons. Once the discussion has dealt with all the topics, and the plan (or plans) is defined, each student should write her own business plan for it. It should include:

\begin{itemize}
    \item Name of project - IncreaseStack.
    \item Main goals and brief description of the start-up.
        \begin{itemize}
            \item 
        \end{itemize}
    \item License type and overall strategy.
    \item Briefly detail some strategic areas.

    \item Fill in Osterwalder’s canvas, in the following (suggested) order (you don’t have to actually write on the poster, but just include information for each of the following topics):
    \begin{itemize}
        \item Clients segments
        \item Value proposal
        \item Channels
        \item Key resources
        \item Cost structure
        \item Revenues streams.
        \item Customer relationships
        \item Key activities
        \item Key partners
    \end{itemize}
\end{itemize}

\section{Specific report}

Minimum: 20 points, maximum: 80 points
Specific report about a certain business model or strategy based on FLOSS, showing its general aspects, but also analyzing companies already putting it into place, discussing advantages and drawbacks of the model, etc. A detailed DAFO analysis has to be a part of the report.

As a result of this activity, the student should produce:
\begin{itemize}
    \item A ‘traditional’ written report.
    \item A video or audio presentation (10 min. maximum).
    \item A set of slides supporting the presentation.
\end{itemize}

It is important to detail all the references, and to heavily root the report on data and/or specific works publicly available. The video or audio presentation wil be uploaded to some audio or video web hosting site which allows for download of the whole audio or video file, not only streaming (eg, blip.tv)

\end{document}
