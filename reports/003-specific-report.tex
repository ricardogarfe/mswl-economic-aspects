\documentclass[11pt]{scrartcl}
\usepackage[parfill]{parskip}
\usepackage{graphicx}
\usepackage{booktabs}
\usepackage{tabulary}
\usepackage{float}

\title{\textbf{Economic Aspects\\}}
\subtitle{Specific Report}
\author{Ricardo Garc\'ia Fern\'andez}
\date{\today}

\begin{document}

\maketitle

\section{Specific report}

Minimum: 20 points, maximum: 80 points
Specific report about a certain business model or strategy based on FLOSS, showing its general aspects, but also analyzing companies already putting it into place, discussing advantages and drawbacks of the model, etc. A detailed DAFO analysis has to be a part of the report.

As a result of this activity, the student should produce:
\begin{itemize}
    \item A ‘traditional’ written report.
    \item A video or audio presentation (10 min. maximum).
    \item A set of slides supporting the presentation.
\end{itemize}

It is important to detail all the references, and to heavily root the report on data and/or specific works publicly available. The video or audio presentation wil be uploaded to some audio or video web hosting site which allows for download of the whole audio or video file, not only streaming (eg, blip.tv)

\section{Title}

\par Me, my own company, or put another way: expertise in software development.

\par How to get experience developing FLOSS (Free Libre Open Source Software) and use it to get customers. What is a customer? for a single person is a client company itself, ie get a job at the company I want.

\par But we must bear in mind that you have to combine work experience and also have the 'extra' to be within a FLOSS development project. This is the extra quality that gives us this business model. Here we will focus on the development of a FLOSS tool by connecting with the community around us and use this as a gateway to the company.

\section{Target Description}

\par To begin we must draw a target, in this case we will make the goal of working in SpringSource\footnote{http://www.springsource.org/}, a division of VMware\footnote{http://www.vmware.com/} after being purchased in 2009.

\par \emph{SpringSource} is an application development framework for enterprise Java created in 2001 by Rod Johnson while he was writing a book or documented best practices with Java. In 2012 Rod Johnson left the company VMWare for joining the board of directors.

\par SpringSource is a \emph{FLOSS} (Free Libre Open Source Software) framework released under the Apache License Version 2.0\footnote{http://www.apache.org/licenses/LICENSE-2.0}. It has become an ecosystem modulated at the many options available for use in satellites framework and modules that can be used independently, this is the main strength of Spring, FLOSS and modularization be offered.

\par This model is based primarily on Technological Innovation, early adopters, and the classic trial and error.

\section{Useless without FLOSS}

\par FLOSS envelops everything, communication, innovation, knowledge sharing and dissemination of knowledge above.

FLOSS allows us to learn, as in this case the working methodology of SpringSource. Being a FLOSS community teaches its methodology and its strengths.

We can see how the community is scattered Spring, on a fairly comprehensive guide tutoring in \emph{Get Started} section\footnote{http://www.springsource.org/get-started} and \emph{Get Involved}\footnote{http://www.springsource.org/get-involved}

\subsection{Get Started}

This section of initiation can find all kinds of information on the Spring project learning divided into these six groups:

\begin{itemize}
    \item Start a Tutorial - Beginners tutorials.
    \item Grab a Code Sample - Examples functional.
    \item Ask a Question (Forums) - Forum, active and important.
    \item Take a Class (Training) - Spring University to learn to code.
    \item Read the Documentation - Section important, not only for reading, but knowing how to use the documentation.
    \item Video Instruction - Videos that show the tools and operation.
\end{itemize}

\par Try to cover different learning methods, readings, examples, videos, questions, online courses, which thus not only have a way of understanding and developers can choose the path that best suits them.

\par It can be defined as a learning community of Spring Framework within the same community Spring. A sub-community where users can learn and become teachers (using the nomenclature of teaching) around a forum that orbit other services communicating with each other.

\par This is our first step, get in touch with the user community the basic application. Because first you have to know that you are doing, what their possibilities and understand from the reasoning. You have to be in contact with the tool and use it. Use communication channels so we should immediately open an account at the forum and combine the first steps in the manual 'get started' with learning how to use the forum.

\subsection{Get Involved}

\par This is where the road begins regular or advanced user.

\par The presentation of this page shows an enthusiastic welcome comments from the community no matter the level of knowledge you have of Spring as this is a group dedicated to learning and sharing knowledge.

\par The structure is further defined by steps staggered with respect to user roles for this initial stage:

\begin{itemize}
    \item Join the conversarion - Encouraged to enter the ecosystem Spring for information everyday.
    \item Help other users (And get help when you need it too) -Use the forums to help or receive help from others and also using social network Stackoverflow\footnote{http://stackoverflow.com/} con los tags \emph{spring} y \emph{spring-mvc}.
    \item Report issues - Invites report bugs or improvements through the issue tracker JIRA\footnote{http://jira.springsource.org/} de Spring.
    \item Track the latest features and test them out - Active use of JIRA to test new features or errors helping the community to solve.
    \item Contribute code - We housed the Spring code repository on GitHub \ footnote {http://github.com/SpringSource} to use the latest version of the source code published for through tickets and Github defined in relation to the JIRA can provide a solution or improvement through your source code and if it solves or meets the requirements, you will be part of the group of users who contribute code to Spring Framework by signing a contract of contributor\footnote{https://support.springsource.com/spring\_committer\_signup} to the project.
    \item Attend (or give) a talk at a local user group -Encourages the developer to attend talks on Spring or to be the speaker of the same talk to your nearby groups of developers, noting that Spring appreciates this task, help promote the proper use of its Framework directly offering his help.
    \item Attend a spring-related conference - As a last point gives us information about the annual conference of Spring which reports and discusses the improvements and / or developments of Spring joining it to be more in touch with the community.
\end{itemize}

\par This section is where we can really see the potential of FLOSS projects regarding proprietary. The community that exists around the project and the importance it gives.

\par The most important points of collaboration in this guide are those that have to help us to achieve our goal. We highlight the facilities offered to interact with them, which private companies can only find their successful cases, the name of an application that have been developed and some pictures taken from an image bank with smiling people working with no nothing to do with business. Where it appears that the stereotype is the predominant guide.

\par However in FLOSS companies like Spring, is a more human, real people who work there as you can get in touch with them through different communication channels that provide and more importantly, they also know the importance these channels. And if there are pictures on the website, at least they themselves or so says wikipedia.


\section{Not a new model}



\section{Advantages and Drawbacks}

\section{SWOT Matrix}



\end{document}
