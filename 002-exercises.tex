%%This is a very basic article template.
%%There is just one section and two subsections.
\documentclass{article}

\begin{document}


\section{Title}


\subsection{Description}

Think about an hypothetical start-up company.
Activity based on or revolving around FLOSS.
Create a short business model plan.

\subsection{Organization}

Groups of up to three persons (for discussing the plan)
Each student must make her own business project.
Name of project. Ponte el gorro.
Goals.
Select license type and strategy.
Briefly detail strategic areas.

\subsection{Osterwalder's model}

Introducción al negocio.

OpenEdu

\begin{itemize}
    \item Clients segments - a quien va dirigido.
    
    \begin{itemize}
        \item Centros educativos (publicos y privados)
        \item Empresas.
        \item Formadores.
        \item Individuos que se conviertan en formadores.
        \item Tener en cuenta: idioma, segmentación, área geográfica,
    \end{itemize}
    
    \item Value proposal - nuevo valor añadido.
    
    \begin{itemize}
        \item Ahorro costes compartido por empresas.
        \item Generar material libre
        \item Red social educativa por sector/campos
        \item Meritocracía
        \item (floqq)
        \item la forma de ofrecer el producto.
    \end{itemize}
    
    \item Channels - canales de distribución, como acceder a los canales.
    
    \begin{itemize}
        \item Internet
        \item Campañas (web marketing)
    \end{itemize}
    
    \item Key resources - recursos clave necesarios: desarrolladores,
    zapatos, FLOSS, contactos.
    
    \begin{itemize}
        \item Gente para dar los cursos.
        \item universidad.
        \item Herramienta colaborativa de formación
        \item Herramienta colaborativa de generación de contenidos
        \item desarrolladores
        \item experto en comunidades
    \end{itemize}
    
    \item Cost structure - Cuanto va a costar producirlo; producto
    (materias primas) o servicio. análisis, sueldos, precios
    (materia prima/venta). Material. Costes relevantes. Si el FLOSS supone un
    ahorro diferencial se ha de mencionar específicamente.
    \begin{itemize}
        \item Desarrollo de la plataforma (CMS Existente a expandir), hosting, marketing,
        \item FLOSS - ahorro de los recursos básicos para el negocio, el software base.
        \item Números - importante.
        \item Open Core - platform provider
    \end{itemize}
    
    \item Revenues streams - Análisis de los flujos de ingresos con respecto al
    software libre. Gastos/Ingresos.
    \begin{itemize}
        \item Recuperar el capital invertido mediante los cursos publicitados de la gente, empresas, etc y su difusión.
    \end{itemize}
    
    
    \item Customer relationships - Cómo acceder a los clientes ? si existen en
    tus contactos, anuncios, etc. Si los incorporo o no al desarrollo del producto
    en la empresa como comunidad. Interactuar con los clientes. Puede que no te
    interese mantener el contacto con tus clientes.
    
    \begin{itemize}
        \item Corrección de ejercicios, examen, meritocracia en cursos aprobados. Fidelización de clientes.
    \end{itemize}
    
    
    \item Key activities - Actividades clave. Clave para el negocio ninguna otra,
    dar a conocer el producto, desarrollarlo, etc.
    \begin{itemize}
        \item Dar a conocer el producto y buen trato con la universidad (generadores de conocimiento)
        \item Negocio en concreto - generar comunidad.
    \end{itemize}
    
    
    \item Key partners - Gente que ayuda a la comercialización, desarrollo.
    Socio es todo aquel que colabora en un negocio, convertir a los clientes en socios.
    \begin{itemize}
        \item universidad, escuelas, empresas, comunidad.
    \end{itemize}
    
    
\end{itemize}

Conclusión de cómo el software libre ha influido en la creación de mi negocio.

\end{document}
