\documentclass[11pt]{scrartcl}

\title{\textbf{Economic Aspects\\
                Exercises}}
\subtitle{Business models}
\author{Jesús M. González-Barahona}
\date{\today}

\begin{document}

\maketitle

\section{Title}

\subsection{Description}

Think about an hypothetical start-up company.
Activity based on or revolving around FLOSS.
Create a short business model plan.

\subsection{Organization}

Groups of up to three persons (for discussing the plan)
Each student must make her own business project.
Name of project. Ponte el gorro.
Goals.
Select license type and strategy.
Briefly detail strategic areas.

\subsection{Osterwalder's model}

Introducción al negocio.

OpenEdu

\begin{itemize}
    \item Clients segments - a quien va dirigido.
    
    \begin{itemize}
        \item Centros educativos (publicos y privados)
        \item Empresas.
        \item Formadores.
        \item Individuos que se conviertan en formadores.
        \item Tener en cuenta: idioma, segmentación, área geográfica,
    \end{itemize}
    
    \item Value proposal - nuevo valor añadido.
    
    \begin{itemize}
        \item Ahorro costes compartido por empresas.
        \item Generar material libre
        \item Red social educativa por sector/campos
        \item Meritocracía
        \item (floqq)
        \item la forma de ofrecer el producto.
    \end{itemize}
    
    \item Channels - canales de distribución, como acceder a los canales.
    
    \begin{itemize}
        \item Internet
        \item Campañas (web marketing)
    \end{itemize}
    
    \item Key resources - recursos clave necesarios: desarrolladores,
    zapatos, FLOSS, contactos.
    
    \begin{itemize}
        \item Gente para dar los cursos.
        \item universidad.
        \item Herramienta colaborativa de formación
        \item Herramienta colaborativa de generación de contenidos
        \item desarrolladores
        \item experto en comunidades
    \end{itemize}
    
    \item Cost structure - Cuanto va a costar producirlo; producto
    (materias primas) o servicio. análisis, sueldos, precios
    (materia prima/venta). Material. Costes relevantes. Si el FLOSS supone un
    ahorro diferencial se ha de mencionar específicamente.
    \begin{itemize}
        \item Desarrollo de la plataforma (CMS Existente a expandir), hosting, marketing,
        \item FLOSS - ahorro de los recursos básicos para el negocio, el software base.
        \item Números - importante.
        \item Open Core - platform provider
    \end{itemize}
    
    \item Revenues streams - Análisis de los flujos de ingresos con respecto al
    software libre. Gastos/Ingresos.
    \begin{itemize}
        \item Recuperar el capital invertido mediante los cursos publicitados de la gente, empresas, etc y su difusión.
    \end{itemize}
    
    
    \item Customer relationships - Cómo acceder a los clientes ? si existen en
    tus contactos, anuncios, etc. Si los incorporo o no al desarrollo del producto
    en la empresa como comunidad. Interactuar con los clientes. Puede que no te
    interese mantener el contacto con tus clientes.
    
    \begin{itemize}
        \item Corrección de ejercicios, examen, meritocracia en cursos aprobados. Fidelización de clientes.
    \end{itemize}
    
    
    \item Key activities - Actividades clave. Clave para el negocio ninguna otra,
    dar a conocer el producto, desarrollarlo, etc.
    \begin{itemize}
        \item Dar a conocer el producto y buen trato con la universidad (generadores de conocimiento)
        \item Negocio en concreto - generar comunidad.
    \end{itemize}
    
    
    \item Key partners - Gente que ayuda a la comercialización, desarrollo.
    Socio es todo aquel que colabora en un negocio, convertir a los clientes en socios.
    \begin{itemize}
        \item universidad, escuelas, empresas, comunidad.
    \end{itemize}
    
    
\end{itemize}

Conclusión de cómo el software libre ha influido en la creación de mi negocio.

\section{Companies running FLOSS business models}

\subsection{Exercise}

Research, using information in the web, the business models of Red Hat, 
Canonical, and Eucalyptus. Describe, briefly, how they earn money, and their 
business strategy. This excercise tryies to be a first approach to business 
models, so no analysis is needed, a brief description is enough.

\subsubsection{Red Hat}

\quotation{ "The source code is free and it's freely available. But we compile
 those bits and we make it enterprise-class."}\footnote{http://arstechnica.com/business/2012/02/how-red-hat-killed-its-core-productand-became-a-billion-dollar-business/}
 
 With this basic principle we can explain Red Hat business model; Enterprise 
 Subscriptions of the product\footnote{http://www.redhat.com/about/subscription/}. 
 This suscription gives to the customer an extra value that doesn't exists
  in dowloadable release, the direct care from RedHat.
 The same code, the same freedom but including its extra value. Also their 
 suscription assistance support (standar and premium) they offer courses,
  training and special packaging of every distro (29.9 dolars).\\
 As a conclusion, they profesionalize their FLOSS product making money with 
 these extra options\footnote{http://blip.tv/linuxcom/jim-whitehurst-explains-red-hat-s-business-model-in-less-than-four-minutes-800398}.

\subsubsection{Canonical}

 Canonical has a closer business model to Red Hat with its distro Ubuntu\footnote{http://www.canonical.com/enterprise-services/ubuntu-advantage/overview}.
 Advantage suscription in Ubuntu provides extra service dedicated to the customer 
 related to specific services included around Ubuntu:
 \quotation{Canonical is to create an ecosystem of products and services around Ubuntu, which would complement the functions of the OS}\footnote{http://www.ghabuntu.com/2009/09/ubuntu-business-model-misunderstood.html}
 Without elaborating another side of their business model is partneship to give
 Canonical/Ubuntu support, so selling parnership to other companies and formation
 to these companies is another business model to standarize Ubuntu methodology.\\
 To conclude, we can say that the business model of the use of FLOSS is located
  around the products developed with this technology, support and expertise.
 
\subsubsection{Eucalyptus}

 Eucalyptus; Open Source Private and Hybrid Clouds from Eucalytus. Here is the 
 slogan from the company, clear on the top of theri website. FLOSS business model
 oriented to the mainstream cloud.
 In section products and services\footnote{http://www.eucalyptus.com/eucalyptus-cloud} 
 we can see what is offer as a product and their partners:
 \begin{itemize}
  \item EUCALYPTUS CLOUD
  \item ECOSYSTEM TOOLS
  \item TRY EUCALYPTUS
  \item EDUCATION
  \item CONSULTING
  \item SUPPORT
\end{itemize}
 Focusing on the support section we can see another sample of FLOSS business model
 as a support service, priority access, specifications and expertise in the product.
 This is very close to other models described above, in another niche but almost the same
 model.
 Partnership is another option for the companies that want approvement seal of 
 Eucalyptus in their products, as in other companies (Canonical and Red Hat) they 
 standarize its proccess to expand their knwoledgement and teach it evolving the
  community around their projects creating its work methodology.
 
\subsection{Conclusion}

As a final conclusion we can say that the business model of each company is 
close to other. Offer extra services to consumers who request them, 
creating standards and partnering with companies that want to export
 the professionalism behind a hallmark of the company itself. 
From what we see as the business model tries through FLOSS free
 base create an ecosystem for the professional who demands 
providing security and reliability by allowing other companies will become 
part of the family of partners. 

\end{document}
