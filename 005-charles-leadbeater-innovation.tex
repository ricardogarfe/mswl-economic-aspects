\documentclass[11pt]{scrartcl}
\usepackage[parfill]{parskip}

\title{\textbf{Economic Aspects\\
                Exercises}}
\subtitle{Charles Leadbeater on innovation}
\author{Ricardo Garc\'ia Fern\'andez}
\date{\today}

\begin{document}

\maketitle

\section{Title}

\section{Description}

View \emph{Charles Leadbeater on innovation}\footnote{http://www.ted.com/talks/charles\_leadbeater\_on\_innovation.html}, and comment about it. In addition to a general comment on the presentation, address the issues detailed in the slides for this assignment.

\subsection{Issues}

\begin{itemize}
	\item View the video
    \item Comment on the aspects that are more related to free software
    \item Select at least one of the following issues (next slides) and comment on it
    \begin{enumerate}
	    \item Pro-ams/prosumers
	    \begin{itemize}
        	\item http://www.clunkers.net/
        	\item The role of pro-ams and prosumers to drive innovation.
        	\item More examples of creativity driven by consumers??
        \end{itemize}
        \item Creativity an innovation sources
        \begin{itemize}
	        \item Big firms have a built-in tendency to reinforce past positive experiences...
	        \item ... but that prevents disruptive innovation to show up.
	        \item http://innocentive.com/
        \end{itemize}
        \item Users can be producers
        \begin{itemize}
	        \item http://en.wikipedia.org/wiki/Shanda
	        \item http://wikifactcheck.org/
	        \item NewsCorp. paywall vs. NYT URL-shortener (social media).
        \end{itemize}
    \end{enumerate}
\end{itemize}

\section{Users can be producers}

\emph{We don't need a organization to be organized.}

\subsection{Needs \& Creativity}

Creativity comes from anywhere, even that incentive comes from the premise of the need, so you have to be next to people who have a need and develops a solution to this. This person will develop an appropriate solution for your problem or improvement.

In this case, draws FLOSS (in a majority) of needs of people working together. It is much more difficult for a group of developers sent by a company, has the same capacity for creativity that one person dedicated to developing a solution, the solution to his problem. Because this solution, it is important for him selfishly.

\subsection{Cathedral and the Bazaar}
\label{sub:cat-baz}

% TBC
We can draw a parallelism with Eric Raymond's essay 'The Cathedral and the Bazaar'.

\subsection{Vuelco a los sistemas de producción}
\label{sub:turnover-production}

\par The traditional means of production are being relegated to second place, due to the ease of solving the needs in developing FLOSS.

Las barreras existentes a la hora de obtener resultados, son ínfimas en FLOSS. Debido a esto, los usuarios van a cubrir sus necesidades ayudando a las necesidades de los demás, aunque no existan, debido a que publican su conocimiento en un lugar común. Este lugar común sería utilizado por la comunidad para gestionar entre todos, mediante unas normas racionales, el conocimiento libre. 

Debido a esto, los sistemas de producción se ven relegados a un segundo plano. Ya no mandan en el mercado impuesto o creado por ellos mismos. El primer plano es el lugar común de conocimiento. Los sistemas de producción se establecen en el mismo nivel que un usuario normal. Horizontaliza los derechos haciendo a todos iguales. 

Las comunidades/ciudanos FLOSS crean ese espacio vacío de compartición de conocimiento.

\subsection{Education}
\label{sub:education}

Usuarios más educados, más profesionales mediante la implicación por necesidad del producto.
Debido a esta afirmación, cuantos más usuarios se impliquen en la toma de decisiones en el espacio vacío, se tomarán mejores decisiones para mejores soluciones.

\begin{thebibliography}{9}

    \bibitem{edu-ciudadania}
    Educación para la ciudadanía - Democracia, Capitalismo y Estado de Derecho,\\
    Carlos Fernández Liria, Pedro Fernández Liria, Luis Alegre Zahonero,\\
    http://www.akal.com/libros/EducaciOn-para-la-CiudadanIa/9788446026136

    \bibitem{charles-innovation}
    Charles Leadbeater on innovation,\\
    http://www.ted.com/talks/charles\_leadbeater\_on\_innovation.html
    
\end{thebibliography}

\end{document}
