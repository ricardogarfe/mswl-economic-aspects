\documentclass[11pt]{scrartcl}
\usepackage[parfill]{parskip}
\usepackage{graphics}
\usepackage{hyperref}

\graphicspath{{images/}}

\title{\textbf{Economic Aspects\\
                Exercises}}
\subtitle{Charles Leadbeater on innovation}
\author{Ricardo Garc\'ia Fern\'andez}
\date{\today}

\begin{document}

\maketitle

\vfill

\begin{flushright}
    \copyright  2013 Ricardo Garc\'ia Fern\'andez - ricardogarfe [at] gmail [dot] com.

    This work is licensed under a Creative Commons Attribution- ShareAlike 3.0 License.
    To view a copy of this license visit:
 
    \url{http://creativecommons.org/licenses/by-sa/3.0/legalcode}.
\end{flushright}

\begin{figure}[h]
    \begin{flushright}	
        \includegraphics{by-sa}
        \label{fig:by-sa}
    \end{flushright}
\end{figure}

\newpage

\section{Title}

\section{Description}

View \emph{Charles Leadbeater on innovation}\footnote{http://www.ted.com/talks/charles\_leadbeater\_on\_innovation.html}, and comment about it. In addition to a general comment on the presentation, address the issues detailed in the slides for this assignment.

\subsection{Issues}

\begin{itemize}
	\item View the video
    \item Comment on the aspects that are more related to free software
    \item Select at least one of the following issues (next slides) and comment on it
    \begin{enumerate}
	    \item Pro-ams/prosumers
	    \begin{itemize}
        	\item http://www.clunkers.net/
        	\item The role of pro-ams and prosumers to drive innovation.
        	\item More examples of creativity driven by consumers??
        \end{itemize}
        \item Creativity an innovation sources
        \begin{itemize}
	        \item Big firms have a built-in tendency to reinforce past positive experiences...
	        \item ... but that prevents disruptive innovation to show up.
	        \item http://innocentive.com/
        \end{itemize}
        \item Users can be producers
        \begin{itemize}
	        \item http://en.wikipedia.org/wiki/Shanda
	        \item http://wikifactcheck.org/
	        \item NewsCorp. paywall vs. NYT URL-shortener (social media).
        \end{itemize}
    \end{enumerate}
\end{itemize}

\section{Users can be producers}

\emph{We don't need a organization to be organized}\cite{charles-innovation}.

\subsection{Needs \& Creativity}

Creativity comes from anywhere, even that incentive comes from the premise of the need, so you have to be next to people who have a need and develops a solution to this. This person will develop an appropriate solution for your problem or improvement.

In this case, draws FLOSS (in a majority) of needs of people working together. It is much more difficult for a group of developers sent by a company, has the same capacity for creativity that one person dedicated to developing a solution, the solution to his problem. Because this solution, it is important for him selfishly.

\subsection{Cathedral and the Bazaar}
\label{sub:cat-baz}

\par We can draw a parallelism with Eric Raymond's essay \emph{The Cathedral and the Bazaar}\cite{cat-baz}.

\par In \emph{The Cathedral and the Bazaar} we see how the organization of the type considered freer Bazaar since growth is from bottom to top. Based on decisions by many people, laying a more dynamic and consensual basis. While in the cathedral model, growth or the way forward, comes from above, from one person to the ground.

\par Therefore, the generation of knowledge and innovation is relegated to a few, like Charles explains in the video with reference to the example in the places reserved for creative, with foosball, sofas, drinks, all wonderful but alienated than really follows its natural course, the user community. This is not to critize to the work done by these people and their brilliant ideas, but do not represent the majority of People, In this case users.

\par The more people involved, the better evolve the business, the community, after all, everything.

\subsection{Turn away from the means of production}
\label{sub:turn-production}

\par The traditional means of production are being relegated to second place, due to the ease of solving the needs in developing FLOSS.

\par Existing barriers to getting results, are negligible in FLOSS.
Because of this, users will meet their needs by helping others, even if they do not exist, because they publish their knowledge in a common place. This commonplace, would be used by the community to manage together, by rational standards, free knowledge.

\par Due of this, means of production are relegated to the background. We do not decide in the market created by themselves. The foreground is the common place of knowledge. Means of production are set to the same level as a normal user. Rights horizontalize making everyone equal.

\par FLOSS communities, composed of citizens, build that empty space knowledge sharing\cite{edu-ciudadania}.

\subsection{Education}
\label{sub:education}

\par Users more educated, more professional by involving partide generated the need through the product.

\par Following this statement, the more users are involved in decision making in empty space, it will make better decisions for better solutions.

\begin{thebibliography}{9}

    \bibitem{edu-ciudadania}
    Educación para la ciudadanía - Democracia, Capitalismo y Estado de Derecho,\\
    Carlos Fernández Liria, Pedro Fernández Liria, Luis Alegre Zahonero,\\
    http://www.akal.com/libros/EducaciOn-para-la-CiudadanIa/9788446026136

    \bibitem{charles-innovation}
    Charles Leadbeater on innovation,\\
    http://www.ted.com/talks/charles\_leadbeater\_on\_innovation.html

    \bibitem{cat-baz}
    The Cathedral and the Bazaar,\\
    Eric Raymon,\\
    http://www.catb.org/esr/writings/homesteading/cathedral-bazaar/
    
\end{thebibliography}

\end{document}
