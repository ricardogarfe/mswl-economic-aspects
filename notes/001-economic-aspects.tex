\documentclass[11pt]{scrartcl}
\usepackage[utf8]{inputenc}
\usepackage[spanish]{babel}

\title{\textbf{Economic Aspects\\
                Lesson 1}}
\subtitle{Introduction and motivation}
\author{Jesús M. González-Barahona}
\date{\today}

\begin{document}

\maketitle

\section{Introduction}

La autogestión del software libre parece ser el punto más importante dentro de la \emph{esperanza de vida} del Software Libre, la sostenibilidad.
Una decisión que nos ofrece un tipo de modelo de negocio desde la diferencia máxima con respecto al privativo e incluso parejo.\\

Puntos a tener en cuenta económicamente:
\begin{itemize}
    \item Impacto de la economía en un nicho o región.
    \item El uso del Software Libre en empresas no tecnológicas.
    \item FLOSS estrategia de empresa. Ejemplo Android-Google.
    \item Open innovation: nuevas formas de innovación, innovación abierta.
\end{itemize}

Modelos de negocio, autogestión, viabilidad.

Como se financia el desarrollo y la mantenibilidad.Empresas, aportes, fundaciones, eccétera.

\section{Evaluación}

\begin{itemize}
    \item Ejercicios del foro.
    \item Entradas del blog.
    \item Apuntes colaborativos.
    \item Informe específico.
\end{itemize}

\section{Introducción y motivación}

Para empezar, servidores web y la evolución de Apache httpd\footnote{http://httpd.apache.org/}.

El control de los servidores es vital para el uso de (exploradores).

\section{El crecimiento del Software Libre}

En estos tres ejemplos podemos ver la evolución del Software Libre o simplemente la inclusión en algunos mercados 'controlados' como derivan en cambios significativos en los mismos e inclusos vuelcos.

\subsection{Servidores Web}

Gráficos relacionados con el Software Libre. Netcraft Surverys\footnote{http://news.netcraft.com/archives/2012/11/01/november-2012-web-server-survey.html}.

En un pequeño análisis podemos ver como Apache se estabiliza como 'dominador' durante estos 20 años en el mundo como servidor web siendo un producto de Software Libre.

La facilidad de adopción de Apache, los pocos impedimentos y el acompañamiento de los estándares dieron una fórmula de éxito abrumadora con respecto a los demás y el paso del tiempo.
La comunidad Apache ha tenido que ir nutriéndose de colaboraciones, a través de todos los medios, personas, empresas, tiempo, al fin y al cabo.
Las ayudas al proyecto como el ejemplo de IBM apostando por el servidor Apache son un ejemplo de obtención de recursos y ganar aceptación hacia los demás al verse respaldado por esta empresa y de esta manera renutre a IBM dotándola de un sevidor Apache sin la necesidad de tantos activos invertidos creando un negocio de servicios alrededor del servidor Apache.

\subsection{Navegadores}

En el mercado de navegadores podemos ver que los cuatro primeros competidores, tres son proyectos de Software Libre y dos de ellos Chrome y Safari vienen del mismo núcleo Webkit\footnote{http://www.netmarketshare.com/}

Para que un mercado sea estable ha de haber un primer competidor con el 60 - 70 por ciento del mercado.

\subsection{Móviles}

La historia de Nokia, Apple y el último jugador en dar la vuelta al mercado, Google.

\subsection{Adopción de Software Libre}

Empresas y el uso de Software Libre, la adopción y su aceptación mediante el conocimiento.
% TODO: Buscar gráfica de empresas que utilicen Software Libre.

En los 500 ordenadores más potentes del mundo, alrededor de 480 tienen Linux\footnote{http://en.wikipedia.org/wiki/TOP500}. Se ha pasado de la pregunta: "¿ cuántos ordenadores del  \emph{TOP500} tienen Linux ?" a "¿Que distribución de Linux es la más usada en la TOP500 ?".

\subsection{Servidores corporativos}

En esta encuesta 'manual' se ve que existen dos actores principales, Windows y Linux. En distintos tipos de empresas varían los resultados.

\subsection{Escritorio}

Terreno absoluto Microsoft.

\subsection{Conclusión}

No se pueden predecir cambios en este tipo de mercados, estos análisis se pueden evaluar a posteriori sin poder predecir como invertir según que cambios.

\section{Un modelo económico diferente}

El tiempo es un recurso finito.

Las teorías económicas tradicionales se basan en, agentes prodcutores y agentes consumidores que se aglutinan en un mercado donde los pone en contacto. Es el modelo económico básico del comercio.

"La parábola del pan" % Explicar

La economía tradicional del mercado no es capaz de explicar el modelo de Software Libre ya que no hay una base de precios tradicional por lo que el paradigma de negocio es distinto al habitual.

Ejemplos parecidos, como ONGs, parques públicos, etc, no son el modelo tradicional ya que el beneficio del invertir el tiempo o el dinero, no es monetario.

\section{Conseguir o atraer recursos}

Vídeo en TED de Yochai Benkler\footnote{}.

Maximizar el redimiento a través del mínimo aporte de recursos por parte de miles de personas, es decir sacar más rendimiento de muchas pequeñas dosis que en pocas grandes dosis.

Estado, empresa y sociedad. Las ordenadores que son las herramientas de las que disponemos cada día nos aportan un 'potencial' que hace 50 años sólo estaba al alcance de grande empresas. Mirando a través de esta perspectiva tenemos en nuestras manos una herramienta muy útil que nos da acceso a la información mediante la comunicación.

Google, genera contribuciones sin que la gente tenga que hacer nada. Mediante esta estrategia, page-rank, revolucionó la cooperación de OpenDirectory y el inicial YahooDirectory.

Estrategia de gente que quiera hacer o que no le importe hacer. El valor está en la gente que los usa, si entras y no ves lo que venden, es que te están vendiendo a ti.

Sistemas económicos tradicionales.\\

% Convertir a tabla en LaTeX
\begin{tabular}{ l l l}
                & Market-based    &   Non-market    \\
Descentralizado & Price-system    &                 \\
Centralizado    & Firm hierarchy  &  Government;Non-profits \\
 \end{tabular}

\emph{Descentralizado basado en el mercado}; compra-venta regulada por el mercado.
\emph{Centralizado basado en el mercado}; economía de empresa.
\emph{Centralizado basado en Gobiernos u ONG}; decisiones centralizadas y utilizar los métodos para el provecho.

\emph{Descentralizado evitando el mercado};

La propiedad en el mundo de la información se multiplexa.

\section{Exercises}

\subsection{True or False statements around  based on this 10 myths}

See Carlo Daffara article \emph{“Ten myths about FLOSS business models”}\footnote{http://www.groklaw.net/articlebasic.php?story=20070828132340846}.

Each one can be right or wrong (or arguable):
\begin{itemize}
    \item FLOSS does not prevent prices from being established for it.\\
        \textbf{True}- FLOSS Software doens't mean it's without charge.
    \item FLOSS licenses aim to suppress any ownership claims, they are hostile to intellectual property rights.\\
        \textbf{False} - Se apoyan en la reclamación de propiedad intelectual por parte del licencidor y se apoya en la legistación de propiedad intelectual.
    \item FLOSS development is mainly driven by ad-hoc altruism and volunteer effort.
        \textbf{False} - More and more companies dedicate an effort to FLOSS projects.
    \item If I release my software to the FLOSS community, thousands of developers will suddenly start working for me, for nothing.\\
        \textbf{False} - 
    \item Many big companies and firms (even outside software development business) use FLOSS.
        \textbf{True} - 
    \item FLOSS is inherently unreliable, and not very well supported.\\
        \textbf{False} - If Google use Linux, Linux is good for your company.
    \item Private companies contribute to FLOSS, since they can get profit in exchange.\\
    \item Widespread(ampliamente) FLOSS projects are invariably funded by large private companies.
    \item FLOSS software has achieved to become a market leader in certain sectors.
    \item If a company leading a FLOSS project is acquired, the new owner can lock the source code and force all its customers to pay for it.
    \item Using GPL you can't make bussiness.
\end{itemize}

\subsection{Imagine...}

You have a product to publish as FLOSS and you have to decide which license would you choose between GPL and Apache and somebody tell that GPL 

Apple Licensed WebKit with Apache version2 to avoid poisoned contributions with GPL.

\begin{thebibliography}{9}

    \bibitem{flossmanuals}\label(sec:flossmanuals)
        FLOSS Manuals,\\
        Generación de libros a partir de HTML,\\
        http://www.flossmanuals.net/

\end{thebibliography}

\end{document}